\section{Εργαλεία Και Τεχνικές}
\subsection{Πλαίσιο Λογισμικού}
Για την υλοποίηση του REST API χρησιμοποιήθηκε το framework\footnotemark\footnotetext{Software framework, a reusable set of libraries or classes for a software system} \textbf{Lumen-Laravel} \cite{lumen-laravel} που ενδείκνυται για το σκοπό αυτό. Ωστόσο τα προαπαιτούμενα για την χρήση του framework είναι:  

\begin{itemize}
\item PHP >= 5.6.4
\item OpenSSL PHP Extension
\item PDO PHP Extension
\item Mbstring PHP Extension
\end{itemize}

τα οποία ήταν διαφορετικά στο τοπικό υπολογιστή μου (PHP 5.61) και για το λόγο αυτό χρησιμοποιήθηκε το εικονικό περιβάλλον \textbf{Laravel Homstead} \cite{laravel-homestead} με τη χρήση VirtualBox\footnotemark\footnotetext{Oracle VM VirtualBox, supports the creation and management of virtual machines} και Vagrant\footnotemark\footnotetext{Open-source software for building and maintaining porable virtual devepment enviroments}.

\subsection{Περιβάλλον Ανάπτυξης}

Το \textbf{Laravel Homstead}\cite{laravel-homestead} είναι ένα επίσημο πακέτο Vagrant box που παρέχει ένα περιβάλλον ανάπτυξης χωρίς να απαιτείται η εγκατάσταση PHP, Web Server και άλλα προγράμματα λογισμικού στο τοπικό υπολογιστή.

Περιλαμβάνει τα εξής λογισμικά:

\begin{itemize}
\item Ubuntu 16.04
\item Git
\item PHP 7.1
\item Nginx
\item MySQL
\item MariaDB
\item Sqlite3
\item Postgres
\item Composer
\item Node (With Yarn, PM2, Bower, Grunt, and Gulp)
\item Redis
\item Memcached
\item Beanstalkd
\end{itemize}

\subsection{Σύστημα Ελέγχου Εκδόσεων}

Για την διαχείριση των εκδόσεων του Rest API και του Web Application χρησιμοποιήθηκε το git\footnotemark\footnotetext{Free sofware, used for software development and other version control tasks} και ένα private repository απο το git-hub\footnotemark\footnotetext{Web-based gii repository hosting service} 